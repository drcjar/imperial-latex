% Define the content of your presentation in this file

% Use example-conference-notes or example-conference-slides to compile
% it to the right output format.
\title{This is the title}
\subtitle{Sub title}
\conference{International Conference}
\date{\today}
\author{James Keirstead}

\usepackage{lipsum}

\begin{document}

\begin{frame}
\maketitle
\end{frame}

\section{Introduction}

\lipsum[1]

\begin{frame}{Again}
Written with Beamer \citep{Tantau_2004} \pause
asds \url{test}

\begin{block}{Test}
This is a block
\end{block}
\end{frame}

\lipsum[2]

\subsection{Lists}

\begin{frame}{Lists}{Subtitle}
\begin{itemize}
\item asdas
\end{itemize}

\begin{description}
\item [test] test
\end{description}

\begin{enumerate}
\item asds
\end{enumerate}
\end{frame}


\lipsum[3]

\begin{frame}{Images}

\includegraphicscopyright[width=2in]{Isambard_Kingdom_Brunel_signature.png}{Source: \href{http://commons.wikimedia.org/wiki/File:Isambard_Kingdom_Brunel_signature.png}{Wikimedia}}

\end{frame}

\lipsum[4]

\begin{frame}[fragile]{Code listing}

% Special rcode environment for R
% Doesn't entirely work
\begin{rcode}
# This is an R comment
x <- 10
\end{rcode}

\begin{lstlisting}[language=R]
# This is an R comment
x <- 10
\end{lstlisting}

% The default listing is Java
\begin{lstlisting}
// A test class
public class Test {
  int x = 10;
}

\end{lstlisting}
\end{frame}


\section*{References and further reading}
Test a reference to \citep{Tantau_2004}.

\begin{frame}[allowframebreaks]{References} 
\scriptsize
\bibliographystyle{plainnat}
\bibliography{references}
\end{frame}


\end{document}
