\documentclass[12pt, a4paper, solutions]{exam}

% Set paper info
\degree{MSc}
\yearpart{}
\extratext{}
\modulenumber{CI9-SD}
\modulename{Sustainable Development}
\examyear{2014}
\examdate{Friday 9 May 2014, 14:00--17:00}
\rubric{Answer \emph{four} of the five questions.  Marks are as indicated.}
\author{Dr James Keirstead}

\begin{document}
\makecover

\begin{questions}[Question]

\item High Speed Two (HS2) is a proposed high-speed rail network connecting London to Birmingham, Manchester, and Leeds.  Construction on the first phase to Birmingham is scheduled to begin in 2017 and open in 2026; if Phase 2 proceeds, it is expected to be completed in 2032.  

To guide its design and operations, the HS2 consortium has adopted a sustainability policy which states their vision:
\begin{quotation}``of a high speed railway network which changes the mode of choice for inter-city journeys, reinvigorates the rail network, supports the economy, creates jobs, reduces carbon emissions and provides reliable travel in a changing climate throughout the 21st century and beyond.''
\end{quotation}

\begin{enumerate}[(a)]
\item State the Brundtland definition of sustainable development.\points{3}

\begin{solution}
The Brundtland Definition: ``development which meets the needs of the present without compromising the ability of future generations to meet their own needs'' (3 for verbatim, $-1$ for slight miswording)
\end{solution}

\item Assuming that the vision stated above is achieved, critically discuss whether HS2 is `sustainable' according to the Brundtland definition. \points{5}

\begin{solution}
There are two things to catch here:
\begin{itemize}
\item Yes, the vision is compatible as it explicitly acknowledges \emph{future} concerns (2). 
\item However the vision takes it for granted that the network meets ``the needs'' of both present and future, as opposed to alternative infrastructure investments.(2)
\item Overall argument and an justified opinion stated. (1)
\end{itemize}
\end{solution}

\end{enumerate}
\end{questions}

\end{document}
